\documentclass{article}
\title{Lab 4: Broadcast Programming}
\author{Christian Garcia}

\usepackage{listings}
\usepackage{color}
\usepackage{xcolor}
\usepackage[none]{hyphenat}

\definecolor{dkgreen}{rgb}{0,0.6,0}
\definecolor{gray}{rgb}{0.5,0.5,0.5}
\definecolor{mauve}{rgb}{0.58,0,0.82}

\lstset{frame=tb,
	language=C++,
	aboveskip=3mm,
	belowskip=3mm,
	showstringspaces=false,
	columns=flexible,
	basicstyle={\small\ttfamily},
	numbers=none,
	numberstyle=\tiny\color{gray},
	keywordstyle=\color{red},
	commentstyle=\color{dkgreen},
	stringstyle=\color{blue},
	identifierstyle=\color{mauve},
	breaklines=true,
	breakatwhitespace=true,
	tabsize=3
}

\begin{document}
	\maketitle
	
	%\vspace{24pt} % 2 lines
	%\vspace{48pt} % 4 lines
	%\vspace{72pt} % 6 lines
	%\vspace{108pt} % 9 lines
	%\vspace{144pt} % 12 lines
	
	\begin{Huge}
		\section{Questions}
	\end{Huge}
	
	\begin{Large}
		\subsection{\textbf{Creating a UDP socket}}
	\end{Large}
	
	\begin{lstlisting}
		if ((new_socket = socket(AF_INET, SOCK_DGRAM, 0)) < 0){
			perror("Socket creation failed!");
			exit(EXIT_FAILURE);
		}	
	\end{lstlisting}
	
	The above code is how you create a UDP socket in C/C++. While it looks eerily familiar to creating a TCP socket, there is a difference. 
	
	\subsubsection{What does SOCK\_DGRAM mean when creating a socket?}
	\vspace{72pt}
	
	
	\begin{Large}
		\subsection{\textbf{Connecting UDP sockets}}
	\end{Large}
	
	\subsubsection{Out of the connection functions shown in the TCP socket assignment, the only one required for a UDP socket is bind(). Why is this?}
	\vspace{72pt}
	\pagebreak
	
	\begin{Large}
		\subsection{\textbf{Reading and Writing Data Over UDP Sockets}}
	\end{Large}
	
	In C/C++, you can only use 2 functions to send and receive data over a UDP socket. These functions are sendto()/recvfrom(). 
	
	\subsubsection{How do sendto() and recvfrom() exchange data over two UDP sockets?}
	\vspace{108pt}
	
	\begin{Large}
		\subsection{\textbf{Creating a Broadcast Application}}
	\end{Large}
	
	\subsubsection{Now that you know the differences between programming a TCP socket and a UDP socket, you will now create a broadcast program that will send the CPU usage percentage over the network in the language of YOUR choice.}
	\textbf{HINT:} Modify your code from the previous labs to make your life less painful in this assignment.
	
	
\end{document}